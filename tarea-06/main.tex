\documentclass{article}
\usepackage[utf8]{inputenc}
\usepackage{color}

\title{Problemas EXPLICIT-REFS}
\author{Sergio Adair López Sánchez}
\date{Octubre de 2022}

\begin{document}

\maketitle

\section{Ejercicios del libro EOPL 3ra ed. sección 4: State}

\textbf{4.8}  Muestra exactamente en qué parte de nuestra implementación del almacenamiento las operaciones toman tiempo lineal en lugar de tiempo constante.

\begin{verbatim}
newref :
    (length the-store)
    (append the-store (list val))

deref :
    (list-ref the-store ref)

setref! :
    (setref-inner
        (cdr store1) (- ref1 1))
        
\end{verbatim}
\textbf{4.9}  Implementa el almacenamiento en tiempo constante representandolo como un vector de Racket. ¿Qué perdemos al usar esta representación?

\end{document}
